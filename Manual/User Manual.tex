\documentclass[10pt,a4paper,oneside]{report}             % Single-side
%\documentclass[12pt,a4paper,twoside,openright]{report}  % Duplex

%\PassOptionsToPackage{chapternumber=Huordinal}{magyar.ldf}
\usepackage{bm}
\usepackage{amsmath}
\usepackage{amssymb}
\usepackage{enumerate}
\usepackage[thmmarks]{ntheorem}
\usepackage{graphics}
\usepackage{epsfig}
\usepackage{listings}
\usepackage{color}
\usepackage{algorithm}
\usepackage{algorithmic}
\usepackage{csquotes}
\usepackage{lastpage}
\usepackage{anysize}
\usepackage{sectsty}
\usepackage{setspace}  % Ettol a tablazatok, abrak, labjegyzetek maradnak 1-es sorkozzel!
\usepackage[hang]{caption}
\usepackage{hyperref}
\usepackage[english]{babel}
\usepackage{fontspec}
\usepackage{textcomp}
\usepackage{subcaption}
\usepackage{titlesec, blindtext, color}
\usepackage{fancyhdr}
\usepackage{lastpage}
\usepackage{todonotes}
\usepackage{graphicx}
\usepackage{tikz}
\usepackage{subcaption}
\usepackage{float}
\usepackage{listings}
\usepackage{xcolor}
\usepackage[sorting=none, backend=biber]{biblatex}
\addbibresource{bib.bib}
\lstloadaspects{formats}

\lstdefineformat{C++}{%
	\{=\newline\string\newline\indent,%
	\}=[;]\newline\noindent\string\newline,%
	\};=\newline\noindent\string\newline,%
	;=[\ ]\string\space}

\definecolor{commentgreen}{RGB}{2,255,10}
\definecolor{eminence}{RGB}{255,150,10}
\definecolor{weborange}{RGB}{255,10,0}
\definecolor{frenchplum}{RGB}{129,20,83}

\setlength{\headheight}{14.5pt}

\lstset {
	language=C++,
	frame=tb,
	tabsize=4,
	showstringspaces=false,
	numbers=left,
	%upquote=true,
	keywordstyle=\color{eminence},
	morestring=[b][\color{weborange}]__,
	morestring=[b][\color{weborange}]",
	basicstyle=\scriptsize\ttfamily, % basic font setting
	stringstyle=\scriptsize\sffamily, 	 			% typewriter type for strings
	showstringspaces=false,  
	emph={int,char,double,float,unsigned,void,bool,short,long,uchar,uchort,uint,ulong,dim3},
	emphstyle={\color{blue}},
	% keyword highlighting
	classoffset=1, % starting new class
	otherkeywords={>,<,-,!,=,~,+,*},
	morekeywords={>,<,-,!,=,~,+,*},
	morecomment=[f][\color{commentgreen}][0]{//},
	commentstyle=\color{commentgreen},
	keywordstyle=\color{eminence},
	classoffset=0,
}

%--------------------------------------------------------------------------------------
% Main variables
%--------------------------------------------------------------------------------------
\newcommand{\vikauthor}{Márton Szemenyei}
\newcommand{\viktitle}{VATEM3}
\newcommand{\vikdept}{}
\newcommand{\vikdoktipus}{User Manual}

%--------------------------------------------------------------------------------------
% Page layout setup
%--------------------------------------------------------------------------------------
% we need to redefine the pagestyle plain
% another possibility is to use the body of this command without \fancypagestyle
% and use \pagestyle{fancy} but in that case the special pages
% (like the ToC, the References, and the Chapter pages)remain in plane style

\pagestyle{plain}
\setlength{\parindent}{0pt} % 
\setlength{\parskip}{8pt plus 3pt minus 3pt} % 
%\setlength{\parindent}{12pt} % 
%\setlength{\parskip}{0pt}    % 

\marginsize{35mm}{25mm}{15mm}{15mm} % anysize package
\setcounter{secnumdepth}{0}
\sectionfont{\large\upshape\bfseries}
\setcounter{secnumdepth}{2}

%--------------------------------------------------------------------------------------
%	Setup hyperref package
%--------------------------------------------------------------------------------------
\hypersetup{
    pdftitle={\viktitle},        % title
    pdfauthor={\vikauthor},    % author
    pdfsubject={\vikdoktipus}, % subject of the document
    pdfcreator={\vikauthor},   % creator of the document
    pdfkeywords={VATEM3, Video Assisted Measurement, User Manual},    % list of keywords
    pdfnewwindow=true,         % links in new window
    colorlinks=true,           % false: boxed links; true: colored links
    linkcolor=black,           % color of internal links
    citecolor=black,           % color of links to bibliography
    filecolor=black,           % color of file links
    urlcolor=black             % color of external links
}

%--------------------------------------------------------------------------------------
%	Some new commands and declarations
%--------------------------------------------------------------------------------------
\newcommand{\code}[1]{{\upshape\ttfamily\scriptsize\indent #1}}

% define references
\newcommand{\figref}[1]{\ref{fig:#1}.}
\renewcommand{\eqref}[1]{(\ref{eq:#1})}
\newcommand{\listref}[1]{\ref{listing:#1}.}
\newcommand{\sectref}[1]{\ref{sect:#1}}
\newcommand{\tabref}[1]{\ref{tab:#1}.}

\DeclareMathOperator*{\argmax}{arg\,max}
%\DeclareMathOperator*[1]{\floor}{arg\,max}
\DeclareMathOperator{\sign}{sgn}
\DeclareMathOperator{\rot}{rot}
\definecolor{lightgray}{rgb}{0.95,0.95,0.95}

\newcommand{\hsp}{\hspace{20pt}}
\titleformat{\chapter}[hang]{\Huge\bfseries}{\thechapter\hsp}{0pt}{\Huge\bfseries}
\titlespacing*{\chapter}{0pt}{0pt}{40pt}


\author{\vikauthor}
\title{\viktitle}
%--------------------------------------------------------------------------------------
%	Setup captions
%--------------------------------------------------------------------------------------
\captionsetup[figure]{
%labelsep=none,
%font={footnotesize,it},
%justification=justified,
width=.9\textwidth,
aboveskip=10pt}

\renewcommand{\captionlabelfont}{\small\bf}
\renewcommand{\captionfont}{\footnotesize\it}

\fancypagestyle{plain}{%
	\fancyhf{}
	\renewcommand{\footrulewidth}{1pt}
	\chead{\viktitle ~- \vikdoktipus}
	\lfoot{\thepage/\pageref{LastPage}} %/\pagetotal
	\rfoot{\leftmark}
}
\pagestyle{plain}

\pagestyle{fancy}
\renewcommand{\chaptermark}[1]{\markboth{\MakeUppercase{#1}}{}}
\fancyhf{}
\chead{\viktitle}
\lfoot{\thepage/\pageref{LastPage}} %/\pagetotal
\rfoot{\leftmark}
\renewcommand{\footrulewidth}{1pt}

%--------------------------------------------------------------------------------------
% Table of contents and the main text
%--------------------------------------------------------------------------------------
\begin{document}

\singlespacing

%--------------------------------------------------------------------------------------
%	The title page
%--------------------------------------------------------------------------------------
\begin{titlepage}
\begin{center}
\includegraphics[width=60mm,keepaspectratio]{../VAM/icons/ikon01.png}\\
\vspace{0.3cm}
\textbf{University of Veterinarian Sciences}\\
%\textmd{Faculty of Electrical Engineering and Informatics}\\
\textmd{\vikdept}\\[5cm]

\vspace{0.4cm}
{\huge \bfseries \viktitle}\\[0.8cm]
\textsc{\Large \vikdoktipus}\\[2cm]
\textsc{\Large \vikauthor}\\[6cm]

\vfill
{\large \today}
\end{center}
\end{titlepage}

%\pagenumbering{arabic}

%--------------------------------------------------------------------------------------
% tartalom, ábra és táblázatjegyzék
%--------------------------------------------------------------------------------------
\singlespacing
\tableofcontents\thispagestyle{fancy}


\chapter{Basic Usage}

menus
 
\section{Projects}

create

open

\subsection{Setting the number of videos}

blah

\section{Adding videos}

add, order remove

\subsection{Camera distance}

Set camera distance from the ground, used for automatic correction later

only on videos 2 and up

\section{Still database}

Create, clear, reset

play videos, lock, dinterlace, arrange

\subsection{Taking snapshots}

pause, take snapshot, id name

Edit and remove stills

\subsection{Setting global etalon}

select, check etalon, set size

set global etalon, this will be used

\subsection{Open images for stills} \label{sec:openStill}

Open still, pair them

\subsection{Complete stills} \label{sec:compStill}

Go to the right point in video, press button

\section{Schemas}

New, import, edit, remove

\subsection{Open images}

open image to assist with creating the schema
only visual no actual effect

\subsection{Points}

what they are, add edit remove

\subsection{Measurements}

distance and angle, add, edit, remove

\subsection{Correction Measurement}

Measurement value from one video used to correct next

Animals not the same distance from camera as the etalon was, leads to inaccuracy

Example: Etalon on top image is 1m hight, cattle is higher

Cattle height is used to correct measurement, so actual measured cattle height is used to adjust meas values on the top image

Note: camera distance from ground needs to be set correctly, assumes etalon is on the ground

Supports one step forward only vid1->vid2, vid2->vid3, vid3->vid4

\section{Measurement}

New, import, edit, remove

Set schema to use

\subsection{Navigation amongst images}

buttons, double click

\subsection{Add points}

If NN enabled, then auto addition for all except etalon

otherwise click on image, erase points

\subsection{Edit points}

Toggle selection mode, click on point, move, remove

Measurement confidence

\section{Output generation}

Select format and hit generate

\subsection{Formats}

raw csv, xlsx and html supported

measurement images saved in all cases
 

\begin{figure}[!htb]
\centering
\begin{subfigure}{\textwidth}
\centering 
\includegraphics[height=4.7cm]{../VAM/Icons/chessboard.png}
\end{subfigure}
\caption[]
{\small  Image.}
\end{figure} 


\chapter{Advanced Usage}

advanced functions

\section{Calibration}

What is used for, done for each camera

Use wizard, print image

\subsection{Take video}

Change distance of calibration plane
Position in image
orientation rel to the camera
don't move too fast, make sure to stop for 0.5-1 secs at every position

\subsection{Perform calibration}

Load video, set settings hit calibrate

Estimating distortion can cause more good than harm if the taken video is not good quality

\section{Automatic still generation}

AI can detect cattle on side and top images

Launch wizard

\subsection{Set detection area}

Select area where motion is detected

Make sure area contains where cattle is moving, but far from where they enter

Examples good bad

\subsection{Detection}

Settings, motion ignore, motion th, conf t, area th

Launch, preff finish, pair and finish stills, see \autoref{sec:openStill, sec:compStill}

\section{Planimetrics}

Export segmentation data

Navigate, clear, export, weight

\subsection{Polygon tool}

Select polygon from method, put down points on the image, ckick draw polygon

Erase mode

\subsection{Auto segmentation}

Select polygon from method, put a rectangle around object

Click set ROI, then click segmentation

\subsection{Brushes and flood fill}

Select brush, set size and click on the image

Set threshold, put down points and click flood fill

\section{Morphometrics}

Export morphometrics data

\subsection{Setup data}

Select points, add

Change order, remove

\subsection{Export data}

click export

\chapter{Troubleshooting}


\section{Error reporting}

Turn on logging

Send issues here, attach log, project, db, schema and measurement if you can



\nocite{*}
\printbibliography[title=Further Information,heading=subbibliography,keyword=0]


\chapter*{Acknowledgements}

Blah

\label{page:last}
\end{document}